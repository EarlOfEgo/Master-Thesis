\chapter{Analysis of the ported \textsc{BaseX} Android version}
\label{cha:analysis}
In the following chapter the migrated \textsc{BaseX} database will be analyzed.
The focus at this part lies on the evaluation and the performance of the library.
First the specification of the test devices is outlined and an estimation is made to show how the performance of the various devices differ.
Therefore different benchmarks has been executed to approximate the distinctive hardware properties and their execution times.
In the next chapter is shown that the migrated database works without errors and failures on the Android mobile platform using a real devices.
For this purpose different unit test has been also migrated to test the functionality.
After evaluation the functionality the performance has been investigated.
To get an overview of the performance an XML benchmark set, which is especially designed to test the performance of XQuery executions, has been used.
\section{Evaluation of the test devices}
\label{sec:evaluation-of-the-test-devices}
For the measurement of the performance two devices are being used.
First, for testing the \textsc{BaseX} mobile version an Android Tablet PC Samsung Galaxy Tab 2 10.1 and for the normal \textsc{BaseX} version a Lenovo Thinkpad laptop have been used.
Their technical specifications can be seen in Table\ref{tab:test-dev-specs}.
\begin {table}[htpb] 
  \centering
\begin {tabular} {|r|r|r|r|r|r|}
  	\hline
	&CPU&RAM&filesystem&operating system\\
	\hline
	Laptop&Intel Core2 Duo CPU L7100&2 Gb&ext4&Arch Linux\\
	&1.2 GHz&&&(3.12.0)\\
	\hline
	Tablet&Dual-core Cortex-A9&1 Gb&ext4&Android\\
	&1 GHz&&&(4.0.3)\\
	\hline
\end {tabular}
\caption {The technical specifications of the two devices, used for benchmark testing.}
\label {tab:test-dev-specs}
\end {table}

Looking at this table it can be seen that the only similarity of both systems is that they are using the same filesystem, which is ext4, which is short for fourth extended filesystem.
The RAM of the laptop is the double amount of the tablet available RAM and both devices have a dual core CPU with around 1 GHz.
Even if the specifications are not so much differ on the first look, their performance varies in some factors.
Therefore a measurement of the input/output (IO) and CPU speed of booth devices has been made in Section~\ref{sec:analyzing-the-test-devicevs} to get a clearer look how the devices differ.


%\section{Evaluating the working \textsc{BaseX} Android port}
%\label{sec:evaluating-the-working-basex-android-port}
%\textbf{TODO}\\
%In this section will be shown that the created \textsc{BaseX} Android port works and fit all requirements.
%For this purpose the given JUnit tests were been migrated to Android too.
%This shows that all functions work like they work on the standard Java version.


\section{Analyzing the execution performance}
\label{sec:analysing-the-execution-performance}
To investigate the performance of the Android version of \textsc{BaseX} the benchmark suite \textit{XMark} has been used.
\textit{XMark} creates random XML files and has a set of twenty predefined queries that can be used to measure the execution performance of an \textit{XQuery} implementation.
It offers the possibility to choose the size of the random XML files, so that the test queries can be executed on different file sizes.
Therefore it is also being used to find the maximum supported size of a \textsc{BaseX} database on Android.


\subsection{The XMark benchmark suite}
\label{subsec:the-xmark-benchmark-suite}
The XMark benchmark suite is used to identify the speed of the \textsc{BaseX} Android version.
Therefore it offers 20 different XQueries which are designed to benchmark a XQuery implementation.
The queries are separated into different categories, where every category targets a specific aspect of query execution.
The XML files that can be generated by using XMark are all containing the same elements and every file is well-formed.
The content of the generated XML files is randomly generated by using the 17000 most frequently used words~\footnote{ignoring stop words} in the plays of Shakespeare~\cite{schmidtxmark}.
The structure of the randomly generated XML files can be seen in Figure~\ref{fig:xmark-file-structure}.
The size of the XML files can be chosen during the generation process of XMark.
For testing \textsc{BaseX} 15 different files were been generated.
The smallest with the size of 100 Kb followed by files with a size increase of also 100 Kb steps.
Hence, the biggest generated file has a size of 1.5 Mb.\\
The different queries can be seen in Table~\ref{tab:xmark-queries}, where the last row identifies the before mentioned category.
These categories are defined as following:
\begin{enumerate}
  \item Category 1
  \item Category 2
\end{enumerate}


\begin{figure}[h]
\begin{center}
\includegraphics[scale=0.42]{images/xmark-file-elements.png} 
\caption{The structure of a generated XMark XML file. Source:\cite{schmidtxmark}}
\label{fig:xmark-file-structure}
\end{center}
\end{figure}

\begin {table}[htpb] 
  \centering
	\begin{tabular}{r|l}
	  \hline
	  1&Return the name of the person with ID 'person0'.\\
	  \hline
	  2&Return the initial increase of all open auctions.\\
	  \hline
	  3&Return the first and current increase of all open auctions whose current\\
	  &increase is at least twice as high as the initial increase.\\
	  \hline
	  4&List the reserves of those open auctions where a certain person issued\\
	  &a bid before another person.\\
	  \hline
	  5&How many sold items cost more than 40.\\
	  \hline
	  6&How many items are listed on all continents?\\
	  \hline
	  7&How many pieces of prose are in our database?\\
	  \hline
	  8&List the names of persons and the number of items they bought.\\
	  &(Joins person, closed$\backslash$\_auction)\\
	  \hline
	  9&List the names of persons and the names of items they bought in Europe.\\
	  &(Joins person$\backslash$\_auction, item)\\
	  \hline
	  10&List all persons according to their interest; use French markup\\
	  &in the result.\\
	  \hline
	  11&For each person, list the number of items currently on sale whose\\
	  &price does not exceed 0.02\% of the person's income.\\
	  \hline
	  12&For each richer-than-average person, list the number of items currently\\
	  &on sale whose price does not exceed 0.02\% of the person's income.\\
	  \hline
	  13&List the names of items registered in Australia along with\\
	  &their description.\\
	  \hline
	  14&Return the names of all items whose description contains the word 'gold'.\\
	  \hline
	  15&Print the keywords in emphasis in annotations of closed auctions.\\
	  \hline
	  16&Return the IDs of those auctions that have one or more keywords\\
	  &in emphasis.\\
	  \hline
	  17&Which persons don't have a homepage?\\
	  \hline
	  18&Convert the currency of the reserve of all open auctions to\\
	  &another currency.\\
	  \hline
	  19&Give an alphabetically ordered list of all items along with their location.\\
	  \hline
	  20&Group customers by their income and output the cardinality of each\\
	  &group.\\
	  \hline
	\end{tabular}
\caption {The XMark queries.}
\label {tab:xmark-queries}
\end {table}




%\begin{figure}[!ht]
%  \input{plot1}
%  \begin{gnuplot}[terminal=pdf, terminaloptions=color, scale=0.9]
%	set title 'Color graphs with gnuplottex'
%	set key left
%	plot [0:3*pi][-2:2] sin(x), cos(x)
%          set title 'Laptop Steps'
%	  set datafile separator ','
%	  set xlabel 'Query'
%	  set ylabel 'Average time in ms(100 executions)'
%	  set xrange [0:21]
%	  set xtics 1,1,20
%	  set logscale y
%	  set grid ytics lt 0 lw 1 lc rgb '#bbbbbb'
%	  set grid xtics lt 0 lw 1 lc rgb '#bbbbbb'
%	  set key samplen 2 spacing .5 font ',8'
%	  show grid
%	  set style fill solid 0.8 border -1
%	  set boxwidth 0.5 relative
%	  plot for [i=1:14] 'benchmarks/basex-steps-laptop-transposed.csv' u (\$0+1):i title ''.i.'00kb' with linespoints
%	\end{gnuplot}              
  \begin{gnuplot}[terminal=pdf, terminaloptions={font "`Arial"'}]
    plot sin(x), cos(x)
  \end{gnuplot}
%\end{figure}

\subsection{Analyzing the test devices}
\label{sec:analyzing-the-test-devicevs}
The two devices that has been used to benchmark \textsc{BaseX} are totally different and not just in the fact that one is laptop and the other a tablet PC.
They differ in many hardware aspects, but there are two different factors, that are interesting to identify a systems speed and are used for the given purpose.
These are the CPU speed and the input/output (IO) speed, where IO speed is a headline for different operations.
Broadly speaking it can be said that the IO speed can be divided into read and write operations.\\
To measure this values the benchmark tool Bonnie++\footnote{\url{http://www.coker.com.au/Bonnie++/}} is used.
This tool executes different operations and measures how many of these operations can be executed in one second.
It tests sequential output, sequential input, random seeks, sequential create and random create.\\
The sequential output represents the write speed of the system, Bonnie++ uses three different methods to measure this value.
First it writes one character after another by using the \textit{putc()} systemcall. 
After this operation it writes whole blocks with the size of 8192 bytes by using the write() systemcall and than call the close() systemcall.
The last test in this category is the rewrite test and differs from the write test, that the file is not closed after writing.
Therefore one block write comply with 8192 \textit{putc} calls and is way more effective and faster than writing single characters.
The next test is to measure the random seek operations per seconds, which means how often the read/write position can be changed.
It also measures the latency which represents the rotation speed of the disk, but this is obsolete because both systems have flash storage and so there is no revolution per minute to measure.
Bonny is written in C++ and is available on most Linux distributions, but to use it on Android it is necessary to build it from its sources by using the, in Chapter~\ref{sec:migration:the-android-project structure} mentioned, Android Native Development Kit (NDK).
This provides a cross compiler which makes it possible to build C++ code for Android.
For the execution of the benchmarks the version 1.96 from Bonnie++ has been used.
The results of these test executions can be seen in table~\ref{tab:bonnie-results-out}.
%\begin {table}
%\begin {tabular} {|r|r|r|r|r|r|r|r|r|r|r|r|r|}
%	\hline
%		&\multicolumn {3} {|c|} {Sequential}&\multicolumn {2} {|c|} {Sequential}&Random&\multicolumn {3} {|c|} {Sequential}&\multicolumn {3} {|c|} {Random}\\
%		&\multicolumn {3} {|c|} {Output}&\multicolumn {2} {|c|} {Input}&Seeks&\multicolumn {3} {|c|} {Create}&\multicolumn {3} {|c|} {Create}\\
%	\hline
%		&K/sec&K/sec&K/sec&K/sec&/sec&/sec&/sec&/sec&/sec&/sec&/sec&/sec\\
%	\hline
%	Laptop&201&62180&23850&907&98239&1161&17134&137752&13032&19442&170028&11774\\
%	\hline
%	Tablet&5&20828&8756&596&23768&475.6&39&361&167&42&391&238\\
%	\hline
%	\hline
%	Factors&40.20&2.99&2.72&1.52&4.13&2.44&439.33&381.58&78.04&462.90&434.85&49.47\\
%	\hline
%\end {tabular}
%\caption {Results of the Bonnie++ benchmarks.}
%\label {tab:bonnie-results}
%\end {table}


\begin {table}[htpb] 
  \centering
\begin {tabular} {|r|r|r|r|r|r|r|}
	\hline
		&\multicolumn {3} {|c|} {Sequential}&\multicolumn {2} {|c|} {Sequential}&\multicolumn {1} {|c|}{Random}\\
		&\multicolumn {3} {|c|} {Output}&\multicolumn {2} {|c|} {Input}&\multicolumn {1} {|c|}{Seeks}\\
	\hline
		&Char&Block&Rewrite&Char&Block&\\
	\hline
	Laptop&201K&62180K&23850K&907K&98239K&1161\\
	\hline
	Tablet&5K&20828K&8756K&596K&23768K&475.6\\
	\hline
	\hline
	Factors&40.20&2.99&2.72&1.52&4.13&2.44\\
	\hline
\end {tabular}
\caption {Results of the Bonnie++ in- and output benchmarks.}
\label {tab:bonnie-results-out}
\end {table}


Bonnie++ also measures the sequential and random create of a file.
Therefore it creates, queries its status and deletes a file by using the Posix systemcalls \textit{creat()}, \textit{stat()} and \textit{unlink()}.
The results of this test can be seen in Table~\ref{tab:bonnie-results-create}
\begin {table}[htpb] 
  \centering
\begin {tabular} {|r|r|r|r|r|r|r|}
	\hline
		&\multicolumn {3} {|c|} {Sequential}&\multicolumn {3} {|c|} {Random}\\
		&\multicolumn {3} {|c|} {Create}&\multicolumn {3} {|c|} {Create}\\
	\hline
		&creat()&stat()&unlink()&creat()&stat()&unlink()\\
	\hline
	Laptop&17134&137752&13032&19442&170028&11774\\
	\hline
	Tablet&39&361&167&42&391&238\\
	\hline
	\hline
	Factors&439.33&381.58&78.04&462.90&434.85&49.47\\
	\hline
\end {tabular}
\caption {Results of the Bonnie++ sequential/random create benchmarks.}
\label {tab:bonnie-results-create}
\end {table}


%\begin {table}
%\begin {tabular} {|r|r|r|r|r|r|r|}
%	\hline
%		&\multicolumn {3} {|c|} {Sequential}&\multicolumn {2} {|c|} {Sequential}&\multicolumn {1} {|c|}{Random}\\
%		&\multicolumn {3} {|c|} {Output}&\multicolumn {2} {|c|} {Input}&\multicolumn {1} {|c|}{Seeks}\\
%	\hline
%		&Char&Block&Rewrite&Char&Block&\\
%	\hline
%	Laptop (per seconds)&201K&62180K&23850K&907K&98239&1161\\
%	\hline
%	Tablet (per seconds)&5K&20828K&8756K&596&23768K&475.6\\
%	\hline
%	\hline
%	Factors&40.20&2.99&2.72&1.52&4.13&2.44\\
%	\hline
%\end {tabular}
%\caption {Results of the Bonnie++ benchmarks.}
%\label {tab:bonnie-results}
%\end {table}


By analyzing these values it can be said that the laptop is overall faster than the tablet PC.
It is well known that IO operations are the most expensive one on Android, so the achieved result is no surprise. 
But with these values it can be said that there are factors which can tell how much faster it is in specific operations.
This factors are impossible to optimize, because they are a hardware constraint and only changing the hardware can improve them, but they are still important to identify the bottlenecks of the mobile \textsc{BaseX} version.\\
Analyzing Table~\ref{tab:bonnie-results-create} and having a look at the factors shows one value which is, compared to the other factors, extremely high.
The sequential write per character is 40 times faster on the laptop than on the tablet, considering this, it is clear to avoid writing by character instead by block.
Writing by using the block mechanism is just three times slower on the tablet.
The sequential reading of a single character is at the tablet just 1.52 times slower than the at the notebook.
Though the factor of the block reading is quite higher than the character reading factor it is sure to use the block reading because it is much faster than character reading.
But by using block reading the factor need also be considered in the \textsc{BaseX} benchmarks later in chapter~\ref{sec:analysing-the-execution-performance}.
\\
Looking at Table~\ref{tab:bonnie-results-create} it can be seen that the sequential/random creation, reading and deleting is very slow on the Android devices compared to the laptop.
Considering this it should be avoided to often create files, because this is very slow.\\
Bonnie++ gives a good overview about how fast the two system handle their IO operations and how they differ in this aspect.
But there is also another factor which affects the speed of the execution of a program.
This is the CPU speed, it is obvious that this also differs on both test devices.
To evaluate the CPU times a Java program has been written which executes four different CPU intensive operations.
First it does the naive factorial of 5000, where naive means that it just iterates till 5000 and multiplies every step to the result.
The second test is a recursive calculation of the 100th Fibonacci number.
The third sorts an ascending ordered array of 10000 items using the bubble-sort algorithm.
This is done because this is the worst case for the bubble-sort algorithm and has a complexity of $\mathcal O(n^2)$.
The last test makes a naive test if the number 666667 is prime or not, by testing to divide the number by every candidate step by step till the candidate is the square root of the number.
All these test are very CPU intensive, because they use only arithmetic operations.
The results of these test can be seen in Table~\ref{tab:cpu-results}.
\begin {table}[htpb] 
  \centering
\begin {tabular} {|l|r|r|r|r|}
	\hline
		&Factorial&Fibonacci&Bubble Sort&Prime Number\\
	\hline
	Laptop&&&&\\
	(avg. on 1000 runs)&98.19 ms&248.63 ms&180.53 ms&15.85 ms\\
	\hline
	Tablet&&&&\\
	(avg. on 1000 runs)&491.15 ms&2336.35&1045.52&76.41\\
	\hline
	\hline
	Factors&5&9.4&5.8&4.8\\
	\hline
\end {tabular}
\caption {Results of the CPU benchmarks.}
\label {tab:cpu-results}
\end {table}


Except the Fibonacci test it can be said that the CPU of the laptop is about five times faster than the one of the laptop.
This is like the different IO parameters a factor which could not be improved and has to be considered in the \textsc{BaseX} benchmarks as well.
These evaluation is for measuring the performance of the two \textsc{BaseX} versions and how to cope with the results achieved by two different platforms on two different devices.
I

\section{Analyzing the resources consumptions}
\label{sec:analysis:analyszing-the-resource-consumption}
\subsection{Analyzing the consumed storage size}
\label{sec:analysis:analyzing-the-consumed-storage-size}
\section{Improving the shown }
\label{sec:improving}
In this section is shown how the found bottlenecks can be improved and the time consumption of some functionalities are made better.
