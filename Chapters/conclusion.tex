\chapter{Summery}
\label{cha:summery}
In this chapter a conclusion about the present thesis and the achieved results and findings is given as well as an overview of possible future work.
\section{Conclusion}
\label{sec:summery:conclusion}
The goal of migrating XML database \textsc{BaseX} to the Android platform has been achieved, as an implementation of an Android library as well as a client server solution.
Only a few features, like the cryptographic function, have not been migrated to the Android platform in order to get a working solution.



The performance of the database operations and the execution of the XMark benchmark queries have been identified and improved by adjusting the source code of \textsc{BaseX} to Android best practices.
This adjustment also illustrates the big difference between the two platforms, even if it is not visible on the first view.
It has also been shown how difficult it is two compare two different platforms, executed on different devices to each other.
%The replacement of the getter and setter methods affected the execution of the Android version in a big factor and the desktop version meaningless, which underlines the statement of the big differences of the two platforms.
The operations are still slow compared to the desktop version of \textsc{BaseX}, but this is mostly a hardware constraint, because of the lack of various resources on mobile devices.
Even with this disadvantage the Android version of \textsc{BaseX} is the first available and working solution to execute the XQuery language, with its FLOWR expressions, on Android devices.
It is also the first XML database that offers the possibility to be used in the Android operating system.
I%It can also be said, that the used XMark benchmark suite is especially designed to measure the performance of an XML database.




With the rapid distribution of Android and the devices which are using it as an operating system, it is just a matter of time when the first need of an XML database with XQuery support is needed.
Also the development of Android and the always increasing available resources of Android devices path the way for the use of \textsc{BaseX}, in the aspect of performance.



\section{Future work}
\label{sec:summery:future-work}
During the work of the present thesis Google released the new Android version 4.4, codename KitKat.
This version offers different improvements in contrast with the previous versions of Android.
Besides the different features that are made to gain a new user experience Google also claims a performance increase with the release of KitKat and an optimization for devices with lower hardware specifications~\cite{google-kitkat}.
In addition to the improvements that Google publishes with the new Android version, there has also been added another feature to the new version.
This is not mentioned in the official Android KitKat website, because it is an early development state.
It is the Android Runtime (Art) which should replace the Dalvik virtual machine in the near future~\cite{android-art}.
The difference between Art and the DVM is that ART is not a virtual machine, it is more a compiler that translates the code of an application into native machine language.
It can be said that it applies the in Section~\ref{sec:migration:comparison-of-the-two-virtual-machines} explained Just in Time compiler to the whole application and not just for the most executed code parts.
This mechanism is called Ahead of Time compiler, which is done after the application has been installed.
This is also the disadvantage of ART, it can take a lot of time to compile the whole application.
Nevertheless the speed improvement ART should bring is, compared to the one time consumption the compile step brings, better than executing the application with Dalvik, because there is no additional virtual machine with an interpreter to execute.
A lot about can not be said, because it is in a very early state and Google has not provided many information about it.
Future work could be to investigate the improvements ART brings and if there are adjustments in the \textsc{BaseX} Android library can be done to achieve better runtime values using ART.
The improvements that has been done in Section~\ref{sec:improving} could be kept, because most of them are being done to optimize the JIT mechanism from Dalvik, which is the principle of ART.\\
On the other side it could be the hardware aspect that can be investigated.
Because of the lack of possibilities during the work on this thesis the \textsc{BaseX} Android library could only be tested on one device.
A lot of new devices with new hardware are released every year which all have different performance abilities.
Future work could be the focusing on finding a good hardware specification which is the minimum requirement to achieve assumable execution times of the \textsc{BaseX} library.
