\chapter{Introduction}
\label{cha:introduction}
In the last five years, since the release of the first Android phone in October 2008, a lot of progress in the field of mobile devices has been made.
Looking at the market share in the beginning of 2009 shows an amount of 2.8\% sold Android phones~\cite{gandhewar2010google}.
Compared to this, five years later, more than 80\% of the sold smart phones are using Android as their operating system.
This is only one aspect that illustrates the triumphal course Android has made the last few years.
In addition to this the number of available applications for Android phones increase also and has reached more than a million in the third quarter of 2013.
Google offers with its android Play Store the possibility for every developer to easy distribute his application, and offers the possibility to reach millions of customers world wide.


According to bla bla mobile everywhere. Android, And data everywhere, so it's clear to combine database and android.. bla bla
only one solution for android, bla bla, sqlite3
no XML database for android, but needed because of bla bla
then 

\section{Motivation}
\label{sec:introduction:motivation}

\section{Overview}
\label{sec:overview}
The present thesis outlines the migration of the native XML database \textsc{BaseX} to the Android operating system.
Therefore, first an outline is given, about the database, the Android operating system and if there exist related work to the topic.
This section also underlines that there is no native XML database available for Android, which emphasizes the purpose of the present thesis.
In Chapter~\ref{sec:migration:porting-basex-to-android} the source and target platforms of \textsc{BaseX} has been analyzed as well as the internal dependencies which \textsc{BaseX} is using.
After this the database has been migrated to the Android platform, a library has been created which provides the usual \textsc{BaseX} operations.
At the end of the chapter the problems and issues, as well as their solutions, during the migration have been illustrated.
In the next chapter the ported version of \textsc{BaseX} has been analyzed with different techniques and the bottlenecks have been identified.
The Android library has been optimized by improving the found bottlenecks.
In the last chapter a summery is given as well as a conclusion and proposals for future work which could pick up where the thesis ends.


