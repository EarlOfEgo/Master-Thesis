\chapter{Outline}
\label{cha:overview}
In the following chapter an overview of the used components is given.
It also outlines the used technologies of the present thesis.
%It also describes the actual state of the art of XML database as well as on overview of available mobile operating system.
First an introduction in the native XML database \textsc{BaseX} is given and then the mobile operating system Android is introduced.
At the end of the chapter it is outlined if there is any related work available and how this is important or impacting the present thesis.


%\section{Overview of native XML Databases}
%\label{sec:overview:overview-of-native-xml-databases}
%Databases are used for storing bla blub, there are several types of database, they differ in their way to store and process data.
%
%\section{Overview of Mobile Operating Systems}
%\label{sec:overview:overview-of-mobile-operating-systems}
%MAYBE

\section{Introduction into \textsc{BaseX}}
\label{sec:overview:introduction-into-basex}
\textsc{BaseX} is an XML database that also offers the possibility to process XPath/XQuery.
BLABLU\dots..TODO

\section{Overview of Android}
\label{sec:overview:overview-of-android}
Android is an operating system developed by Google and first released in October 2008.~\cite{developers2011android}
It is mostly open source and available in version 4.4.
BLABLUB\dots..TODO
\section{Related Work}
\label{sec:overview:related-work}
Android offers an SQLite database which can be accessed from inside an application by using Java libraries.
SQLite is a relational database system especially designed for embedded devices with a low requirement of resources.
It provides beneath all relevant SQL commands also other mechanisms like views or triggers.~\cite{owens2006definitive}
With its low hardware requirement it is a perfect relative database for the Android operating system.
But it is also the only database available for Android application developer, provided by the operating system.
According to \cite{lamb2010berkleydb} there is an Android implementation of the BerkleyDB database, but it appears not very common, because this is the only reference found to this topic.
Unlike SQLite the BerkleyDB is not a relational database, but it also aims to be deployed on embedded devices.
Another difference between both databases is that SQLite is implemented in C and the Android version of BerkleyDB uses Java as programming language.
Lamb~\cite{lamb2010berkleydb} claims this the reason that the BerkleyDB is three times faster than the SQLite database, because there is no need to translate the Java commands to the native C library, using the Java Native Interface (JNI).
Another approach, for providing an alternative to SQLite, is the object-oriented database Perst~\footnote{\url{http://www.mcobject.com/android}}, which also offers an Android port.
Similar to the BerkleyDB Android version Perst is also implemented using the programming language Java and it targets the embedded devices sector, too.
According to the benchmarks of~\cite{perst-bench} Perst is up to sixteen times faster than the SQLite database.
For an application developer this could be a considerable alternative to the native SQLite database.
Despite this statement those three databases are the only available ones for Android and with the BerkleyDB and Perst not natively covered by Android, most applications are using the standard SQLite Android database for storing their data.
Even if the three databases are all using different database models, an XML database is not available for Android, a port of \textsc{BaseX} to this platform would cover this gap and also provide another alternative to the available databases.\\
\textsc{BaseX} is not only a database it also provides a XQuery processor to query XML documents.
Android offers the possibility to work with XML documents for its applications developers.
Therefore the Document Object Model (DOM) package, as a library, is available.
DOM is an application programming interface (API) for XML documents which have to be well-formed, as well as for HTML documents.~\cite{nicol2001document}
As a result of this, it is possible to parse XML document into documents represented as objects inside an Android application and access the nodes and attributes inside of this document.
Android also provides an alternative to DOM XML parsing, the Simple API for XML (SAX) package which offers, like the name suggests, classes to parse and process XML documents.
Both XML processing mechanisms, provided by Android, have their advantages and disadvantages.
It can be said that one advantage of DOM is that it is possible to access the documents after parsing them, because they are stored as objects inside the application heap.
In contrast to this SAX creates events, or callbacks while reading an XML document.~\cite{roy2001xml}
The price for the storing the objects inside the memory is the decrease of execution time and the increase of resource consumption.~\cite{chenextracting}
Since Android API version 8 the XML Path Language (XPath) is available to query and process the content of XML documents.
Even if there are two possible ways to parse and one to query and process XML documents on Android, there are not the FLOWR expressions which are provided by XQuery available.\\
Besides \textsc{BaseX} there exists other XQuery processor implementations, MXquery~\footnote{\url{http://mxquery.org}} is one that also provides an Android library of its implementation.
This implementation is presented on the Internet~\footnote{\url{https://sites.google.com/site/mxqueryandroid/}} and supports XQuery 3.
There is no research to this implementation available for MXquery on Android and it seems that the development has stopped in 2011.\\
This leads to the result that field of native XML databases, supporting the newest XQuery, implemented and used on Android is not rudimentary explored.
This also applies to the available database systems usable on Android.
Thinking about the amount of alternatives, only the BerklyDB Android implementation, to SQLite this is also a field that could be more opened up.
Migrating \textsc{BaseX} to the Android platform offers hereby a new database system, and an XML query processor supporting the newest XQuery.

